\documentclass[11pt]{article}

\usepackage{hyperref,fontspec,url,xeCJK,geometry}
\usepackage[british,francais]{babel}
\setmainfont{Junicode}
\setCJKmainfont{教育部標準楷書}

\title{Manuel d'installation -- \foreignlanguage{british}{Getting Started}}

\begin{document}

\maketitle
\setcounter{tocdepth}{4}
\tableofcontents

\begin{abstract}
Ce petit manuel indique la marche à suivre pour être capable de reprendre ultérieurement le développement de ce projet. La version la plus à jour est en \texttt{*.pdf} mais un fichier \texttt{*.md}\footnote{Généré par commande \texttt{pandoc -s docs/Installation.tex -o docs/Installation.md}.} est généré pour une lecture directement sur GitHub.
\end{abstract}

\paragraph{Copie du dépôt} Il suffit, sous \textsc{gnu} / Linux, d'exécuter la commande :
\begin{center}\texttt{git clone https://github.com/2marcn/AgentsAutonomes AgentsAutonomes}\end{center}
pour obtenir la dernière version de notre projet dans le dossier \texttt{AgentsAutonomes} (le dernier argument). L'utilisateur d'un système Windows peut obtenir le même résultat en utilisant \href{https://windows.github.com/}{GitHub for Windows}.

\paragraph{Téléchargement d'Activator}

Activator est un ensemble organisé d'outils de programmation et d'exécution que nous utilisons. Il se télécharge sur cette page : \begin{center}\url{https://www.typesafe.com/get-started}.\end{center}Il suffit d'extraire de l'archive le script \texttt{activator}, le fichier de commande \texttt{*.bat} et l'exécutable \texttt{activator-launch-*.*.*.jar} dans le répertoire du projet.

Notre dépôt contient déjà un projet Play, il n'est donc pas besoin d'en créer un. Il contient également Activator qui est directement invoquable sous Windows ou \textsc{gnu} / Linux depuis le répertoire du projet avec la commande : 
\begin{center}\texttt{./activator}\end{center} On entre alors en mode interactif et l'on peut lancer successivement \texttt{clean}, \texttt{compile} puis \texttt{run}. La première phase de compilation peut être particulièrement lente si le répertoire \texttt{~/.ivy2} n'existe pas et qu'Activator doit télécharger toutes les bibliothèques. La commande \texttt{run}, qui lance le serveur Play, prend également en charge la phase de compilation si un changement est détecté.

\paragraph{Modification du projet avec Eclipse}

Pour modifier notre code dans Eclipse, il faut lancer la commande
\begin{center}\texttt{activator eclipse}.\end{center}
Il est toutefois possible d'utiliser un autre \textsc{ide} comme Netbeans, à condition de créer un projet Java à partir de sources existantes.

\paragraph{Déboguage du projet}

On ne peut pas lancer notre programme directement depuis Eclipse puisqu'il faut laisser Activator créer et faire tourner le serveur Play. Néanmoins, il est possible de spécifier à la machine virtuelle Java un numéro de port de débug, par exemple :
\begin{center}\texttt{activator run -jvm-debug 8888}\end{center} puis de lancer sur Eclipse le déboguage d'une application Java distante.

\paragraph{Lancement du programme}

Comme vu précédemment, la commande
\begin{center}\texttt{activator run}\end{center}
permet de lancer notre programme sans activer le déboguage. Il fois le serveur Play initialisé, il faut se rendre à l'adresse
\begin{center}\texttt{localhost:9000}\end{center}
Selon les versions de logiciels utilisés, cette adresse est équivalente à \texttt{/0:0:0:0:0:0:0:0:9000} (\textsc{ip} v.6) ou \texttt{127.0.0.1:9000} en \textsc{ip} v.4. Il est possible que la base de données ait besoin d'être mise à jour : dans ce cas, un script est proposé à l'exécution au démarrage du programme.

\paragraph{Accès à la base de données}

Nous utilisons le moteur de base de données h2 qui a l'avantage d'être léger et accessible facilement avec Activator depuis une interface web. Il faut d'abord lancer Activator :
\begin{center}\texttt{activator}\end{center}
avant de lancer le moteur d'exploration de la base :
\begin{center}\texttt{h2-browser}\end{center}
Si ce n'est pas fait automatiquement, on se rendra à l'adresse : \texttt{http://localhost:8082}. En cas de problème de connexion, il faut bien vérifier que la classe du pilote est \texttt{} et que l'\textsc{url} de \texttt{jdbc} est \texttt{jdbc:h2:mem:play}, comme spécifié dans le \texttt{fichier conf/application.conf}.
\end{document}